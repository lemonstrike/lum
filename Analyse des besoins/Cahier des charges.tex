\documentclass[11pt]{article}

\usepackage[french]{babel}

\title{\textbf{Cahier des charges}}
\author{Fabien BUISSON, Ravi PACHY}
\date{\today}
\begin{document}

\maketitle

\section{Fonctionalités}

\subsection{Site}
\label{sub:site}

\subsubsection{Utilisateurs}

Le site doit pouvoir gérer plusieurs types d'utilisateurs.
Un utilisateur est soit :
\begin{enumerate}
  \item admin global
  \item structure
  
  \item personne
\end{enumerate}

\subsubsection{Relations}

Voir classeur Relations

{\bf Ami}\\[0.6em]

{\bf Abonné}\\[0.6em]
{\bf Membre}\\[0.6em]
{\bf Organisateur}\\[0.6em]
{\bf Dirigeant}\\[0.6em]

\subsubsection{Divers}
Le site doit proposer un moyen simple d'être traduit dans plusieurs langues.

\subsection{Admin global}


% subsection  (end)


\subsection{Fonctionalités communes (Structures/Personnes)}

\subsubsection{Agenda}
\label{ssub:}

Tout utilisateur a à sa disposition un agenda modifiable dans lequel il peut créer des évènements.
L'agenda est composé des évènements de l'utilisateurs et ceux organisés par les amis et structures  auxquelles on est soit membre soit abonné.

\subsubsection{Évènements}

Les évènements peuvent êtres publiés sur la page de l'utilisateur.


\subsubsection{Articles}
\label{ssub:}

% subsubsection  (end)
Tout utilisateur doit pouvoir publier des articles et des évènements datés.
L'utilisateur doit pouvoir gérer un calendrier personnel.
Un utilisateur a une page personnelle qu'il peut gérer.

\subsection{Personne}

Une page personnelle contient les éléments publiés par l'utilisateur (c'est un accueil).
L'acceuil affiche les publications :
  - des structures auxquelles on est membre ou auxquelles on est "abonné".
  - des membres "amis"
L'utilisateur doit pouvoir gérer le niveau de confidentialité de toute information le concernant.

\subsection{Structure}

Une structure est un utilisateur mais :
  - il est amis uniquement avec d'autres



\end{document}
