\documentclass[11pt]{report}

\usepackage[french]{babel}
\usepackage[utf8]{inputenc}
\usepackage{tabularx}
\usepackage[scale=0.8]{geometry}
\title{\textbf{Cahier des charges}}
\author{SBOUB}
\date{\today}
\setcounter{tocdepth}{3}     % Dans la table des matieres
\begin{document}
\maketitle

\tableofcontents

\chapter{Statuts}

L'intégralité du contenu (dossier de conception, codes sources) est sous licence gpl, lgpl ou toute licence libre hormis le contenu ajouté par l'utilisateur qui est sous son entière responsabilité.\\
Chaque utilisateur à le droit de contrôle, modification, suppression de toute information ou contenu le concernant directement.

Toute suppression de données est définitive.


%\chapter{Fonctionalités}

%\section{Site}
%\label{sub:site}

%\subsection{Utilisateurs}

%Le site doit pouvoir gérer plusieurs types d'utilisateurs.
%Un utilisateur est soit : \\

%{\bf Admin global}\\[0.6em]
%L'administrateur général du site.

%{\bf Structure}\\[0.6em]

%{\bf Personne}\\[0.6em]

%\subsection{Relations}

%\begin{table}[h]
%\center
%\begin{tabularx}{15cm}{|r|X|X|X|}
%	\hline
%	& Admin Global & Structure & Personne \\
%	\hline
%	Admin Global & - & - & - \\
%	\hline
%	Structure & - & Abonné & Abonné $<$ Organisateur $<=$ Dirigeant \\
%	\hline
%	Personne & - & Abonné $<$  Organisateur $<=$ Dirigeant & Ami \\
%	\hline 
%\end{tabularx}
%\caption{Relations entre les différents types d'utilisateurs}
%\end{table}


%{\bf Ami}\\[0.6em]
%L'amitié est réservée à deux personnes.La relation d'amitié permet de recevoir et de partager les publications d'un ami (=personne).
%Une publication peut être partagée à un sous-ensemble choisi d'ami.\\

%{\bf Abonné}\\[0.6em]
%Un abonné est une personne ou une structure qui souhaite recevoir certaines publications d'une autre structure.\\

%{\bf Organisateur}\\[0.6em]
%Un organisateur est une personne qui joue un rôle au sein d'une structure.
%Une personne peut faire partie de l'organisation de plusieurs structures.\\

%{\bf Dirigeant}\\[0.6em]
%Le dirigeant est une personne qui occupe une place unique au sein d'une structure.
%Une personne peut être dirigeant de plusieurs structures.

%\subsection{Divers}
%Le site doit proposer un moyen simple d'être traduit dans plusieurs langues.

%\section{Admin global}


%% subsection  (end)


%\section{Fonctionalités communes (Structures/Personnes)}

%\subsection{Agenda}
%\label{ssub:}

%Tout utilisateur a à sa disposition un agenda modifiable dans lequel il peut créer des évènements.
%L'agenda est composé des évènements de l'utilisateurs et ceux organisés par les amis et structures  auxquelles on est soit membre soit abonné.

%\subsection{Évènements}

%Les évènements peuvent êtres publiés sur la page de l'utilisateur.


%\subsection{Articles}
%\label{ssub:}

%% subsubsection  (end)
%Tout utilisateur doit pouvoir publier des articles et des évènements datés.
%L'utilisateur doit pouvoir gérer un calendrier personnel.
%Un utilisateur a une page personnelle qu'il peut gérer.

%\section{Personne}

%Une page personnelle contient les éléments publiés par l'utilisateur (c'est un accueil).
%L'acceuil affiche les publications :
%  - des structures auxquelles on est membre ou auxquelles on est "abonné".
%  - des membres "amis"
%L'utilisateur doit pouvoir gérer le niveau de confidentialité de toute information le concernant.

%\section{Structure}

%Une structure est un utilisateur mais :
%  - il est amis uniquement avec d'autres













\chapter{Fonctionnalités}

	\section{Plugins}
	\section{Publication}
	 % Une publi c'est un mix du contenu des plugins
	\section{Répartition des fonctionnalités}

\chapter{Plugin}
	\section{Définition}
		% Un nom, des types, des tags
	  % c'est normé, ça produit de la publication
	\section{Base}
		\subsection{Editeur WYSIWYG}
		\subsection{Formulaire}
		\subsection{Média}
			\subsection{Photo}
			\subsection{Video}
			\subsection{Fichiers}	
		\subsection{Agenda}
	\section{Tiers}
		\subsection{Afficheur de chromosome}

\chapter{Tags}

	\section{Le rôle}
	% C'est associé à une publication
	

\chapter{Utilisateurs}
	
	\section{Administrateur global}
		\subsection{Rôle}
		\subsection{Droits}
	\section{Personne}
		\subsection{Rôle}
		\subsection{Droits}
			% Une personne a le droit de filtrer ce qu'il recoit
	\section{Structure}
		\subsection{Rôle}
			% ça représente une entité associative/administrative
			% capable de partager du contenu relatif à son rôle réel (irl)%
		\subsection{Gestion}
			% Ya un compte associé à la structure
			% Ya un "dirigeant" qui est une personne et qui a tous les droits de la structure
		\subsection{Droits}
			% creer des "groupes" internes
			% Une structure decide de la portée de ses publications












































\end{document}
